\documentclass[a4paper]{amsproc}
\renewcommand\labelenumi{(\roman{enumi})}
\renewcommand\theenumi\labelenumi
\title{An Atomic Topos for Probability Theory}

\usepackage[english]{babel}
\usepackage{amsmath}
\usepackage{amssymb}
\usepackage{amsthm}
\usepackage{fontspec}
\usepackage[hcentering,bindingoffset=0mm]{geometry}
\usepackage{tikz}
\usepackage{lscape}
\usepackage{mathtools}
% for clickable references
\usepackage{hyperref}\urlstyle{rm}
\usetikzlibrary{cd}

% I like how the sans serif doesn't stand out as much in this font.
\setmainfont{FreeSerif}

\theoremstyle{plain}
 \newtheorem{theorem}{Theorem}[section]
 \newtheorem{proposition}[theorem]{Proposition}
 \newtheorem{lemma}[theorem]{Lemma}
 \newtheorem{corollary}[theorem]{Corollary}
\theoremstyle{definition}
 \newtheorem{example}[theorem]{Example}
 \newtheorem{definition}[theorem]{Definition}
 \newtheorem{notation}[theorem]{Notation}
\theoremstyle{remark}
 \newtheorem{remark}[theorem]{Remark}
 \numberwithin{equation}{section}

% Some shortcuts
% \newcommand{\ob}[1]{ob(#1)}
\newcommand{\id}{\textup{id}}

\DeclareMathOperator{\Hom}{Hom}
\DeclareMathOperator{\Sh}{Sh}
\newcommand{\y}{\textit{y}}
% The star indicates that subscripts are to be typeset below the operator
% instead of to the right
\DeclareMathOperator*{\limit}{lim}
\DeclareMathOperator*{\colim}{colim}
\DeclareMathOperator{\Lan}{Lan}
\DeclareMathOperator{\eq}{eq}
\newcommand{\s}{\textsf{ }}

\newcommand{\Set}{\textup{\textsf{Set}}}
\newcommand{\FinProb}{\textup{\textsf{FinProb}}}
\newcommand{\BPAlg}{\textup{\textsf{BPAlg}}}
\newcommand{\Ind}{\textsf{Ind}}
\newcommand{\C}{\mathcal{C}}
\newcommand{\Prob}{\mathfrak{Prob}}

\newcommand{\la}{\langle\,}
\newcommand{\ra}{\,\rangle}

\newcommand{\ldoub}{[\![}
\newcommand{\rdoub}{]\!]}
\makeatletter
\newcommand{\rmnum}[1]{\romannumeral #1}
\newcommand{\Rmnum}[1]{\expandafter\@slowromancap\romannumeral #1@}
\makeatother

\author{Adam Dauser, Adrian Marti}

\date{}
\begin{document}

\begin{abstract}
WIP
\end{abstract}

\maketitle

\tableofcontents


\section{Introduction}

WIP

\section{Conventions} \label{conventions}

For a small category $\C$, we will denote the Yoneda embedding by $\y: \C \to \Set^{\C^{op}}$. For a morphism $f$ in $\C$, we will denote $\y(f)$ by $f_*$.

We will call a diagram
\[
\begin{tikzcd}
\alpha \arrow[r, "f"] & \beta \arrow[r, "g", shift left] \arrow[r, "h"', shift right] & \gamma
\end{tikzcd}
\]
in a category $\C$ a fork if $fg = hg$.

\section{A combinatorial construction of $\Prob$}

In this section, we define the topos $\Prob$ as a category of sheaves on the category of finite probability spaces. We then proceed by proving properties about this construction, which we will need throughout the paper.


\begin{definition}
    \mbox{}
    \begin{enumerate}
        \item A finite probability space is a finite set $X$ equipped with a map (called probability measure) $\mu: X \to \mathbb{R}_{>0}$ such that \\ $\sum_{x \in X} \mu(x) = 1$.
        \item For a finite probability space $(X,\mu)$ and a subset $a \subset X$, we define $\mu(a) := \sum_{x \in a} \mu(x)$.
        \item A map of finite probability spaces $(X,\mu) \to (Y,\nu)$ is defined as a map of finite sets $f: X \to Y$ that is measure-preserving, i.e. $\mu(f^{-1}(y)) = \nu(y)$. Denote the category of finite probability spaces by $\FinProb$.
    \end{enumerate}
\end{definition}

Notice that all maps of finite probability spaces are surjective. Also notice that we do not allow elements of measure zero.

We want to consider the sheaf topos $\Sh(\FinProb, J_{at})$, where $J_{at}$ denotes the atomic topology. Recall that the atomic topology is defined by letting every nonempty sieve be a covering sieve. We need to check that this yields a Grothendieck topology.

\begin{proposition} \label{pullback_measure}
    Given a diagram of finite probability spaces
    \[
    \begin{tikzcd}
                                & {(X_2,\mu_2)} \arrow[d, "g"] \\
    {(X_1, \mu_1)} \arrow[r, "f"'] & {(Y,\mu)}
    \end{tikzcd},
    \]
    the pullback of the finite sets carries a probability measure $\nu(x,y) := \frac{\mu_1(x) \mu_2(y)}{\mu(f x)}$ such that the diagram
    \[
    \begin{tikzcd}
    {(X_1 \times_Y X_2, \nu)} \arrow[d, "p_1"'] \arrow[r, "p_2"] & {(X_2,\mu_2)} \arrow[d, "g"] \\
    {(X_1, \mu_1)} \arrow[r, "f"']                               & {(Y,\mu)}
    \end{tikzcd}
    \]
    commutes.

    In particular, $J_{at}$ is a Grothendieck topology on $\FinProb$.
\end{proposition}
\begin{proof}
    The only thing to show is that the projections are measure-preserving, since that also implies that $\nu$ is actually a measure. So let $x \in X_1$ and $u := f(x)$.
    \begin{align*}
        \sum_{(x,y) \in p_1^{-1} x} \nu(x,y) &= \sum_{y \in g^{-1} u} \frac{\mu_1(x)\mu_2(y)}{\mu(u)} \\
        &= \frac{\mu_1(x)}{\mu(u)} \sum_{y \in g^{-1} u} \mu_2(y) \\
        &= \frac{\mu_2(x)}{\mu(u)} \mu(u) \\
        &= \mu_2(x)
    \end{align*}
    Similarly, $p_2$ is measure-preserving.

    For the second part, notice that we just proved the \emph{right Ore condition} for $\FinProb$. Checking the Grothendieck topology axioms is a known straightforward formal consequence of this.
\end{proof}

We now introduce the protagonist of this paper.

\begin{definition}
    Define the topos $\Prob$ to be $\Sh(\FinProb, J_{at})$.
\end{definition}

Much of this article is dedicated to finding other equivalent ways of describing $\Prob$. When we wish to leave the particular description of $\Prob$ vague, we will use the notation $\Prob$.

Lemma 2 in section \Rmnum{3}.4 of \cite{maclane2012sheaves} gives us a criterion for checking whether a presheaf is actually a sheaf over the atomic topology:

A presheaf $\mathcal{F}$ on $\FinProb$ is a sheaf over the atomic topology if and only if one can check the following for any morphism $f: \alpha \to \beta$ and any $a \in \mathcal{F}(\alpha)$: If for all diagrams

\[
\begin{tikzcd}
\gamma \arrow[r, "g", shift left] \arrow[r, "h"', shift right] & \alpha \arrow[r, "f"] & \beta
\end{tikzcd}
\]
with $f g = f h$, we have that $\mathcal{F}(g)(a) = \mathcal{F}(h)(a)$, then there is a unique lift $b \in \mathcal{F}(\beta)$ with the property that $\mathcal{F}(f)(b) = a$.

In some cases, one can prove the following condition that is much stronger than being a sheaf. It is especially useful when the sheaf doesn't really use the measure in its definition.

\begin{proposition} \label{sheaf_condition_weak}
Let $\mathcal{F}$ be a presheaf on $\FinProb$. If for all morphisms $f: (X,\mu) \to (Y,\nu)$ the fork
\[
\begin{tikzcd} \mathcal{F} (Y,\nu) \arrow[r, "\mathcal{F}f"] & \mathcal{F} (X,\mu) \arrow[r, "\pi_1", shift left] \arrow[r, "\pi_2"', shift right] & \mathcal{F} (X \times_f X,\bar{\mu})
\end{tikzcd}
\]
is an equalizer, where $\bar{\mu}$ is as in \ref{pullback_measure}, then $\mathcal{F}$ is a sheaf.
\end{proposition}
\begin{proof}
This is the condition from \cite{maclane2012sheaves} described above, but weakened by choosing $\gamma = (X \times_f X,\bar{\mu})$.
\end{proof}


We can now check the sheaf condition for the vast majority of sheaves that we are interested in:

\begin{proposition}\label{subcanonical}
    \mbox{}
    \begin{enumerate}
        \item All morphisms in $\FinProb$ are regular epimorphisms.
        \item The representable functors are sheaves, i.e. $J_{at}$ is a subcanonical topology on $\FinProb$.
        \item All coproducts of representable functors are sheaves.
    \end{enumerate}
\end{proposition}
\begin{proof}
    \mbox{}
    \begin{enumerate}
        \item We need to show that every morphism $f:(X,\mu)\rightarrow (Y, \nu)$ comes from a coequalizer. Our construction of the measure on the pullback in \ref{pullback_measure} gives us the following fork:
        \[
        \begin{tikzcd}[column sep=large]
        (X \times_Y X, \bar{\mu}) \arrow[r, shift left=2] \arrow[r, shift right] & (X,\mu) \arrow[r, "f", two heads] & (Y,\nu)
        \end{tikzcd}
        \]
        Since this is clearly a coequalizer on the underlying sets, the only thing to check is that the uniquely induced maps are measure-preserving. This is straightforward to verify.

        \item Checking that the representable functors are sheaves by using \ref{sheaf_condition_weak} reduces to the fact that the fork above is a coequalizer diagram.

        \item We check the sheaf condition \ref{sheaf_condition_weak} for a coproduct of representables. But since we already know this sheaf condition holds for representables, we merely need to note that equalizers of coproducts are coproducts of equalizers in presheaf topoi (since this fact can be checked pointwise in set).
    \end{enumerate}
\end{proof}

Note that the second and third part are merely formal consequences of the first one. Any category (satisfying the \emph{right Ore condition}) such that the first part holds, also satisfies the second and the third part.

The representable functors are atoms in $\Prob$, i.e. objects with precisely two subobjects: the initial object and the object itself. A crucial property of an atomic topos (a topos of sheaves over a category equipped with the atomic topology) is that every object is a coproduct of atoms. So the natural question arises: Are there more atoms than the representables? In \ref{atoms} we will give a complete list of the atoms and see that the representables are not all of them.


\subsection{Finite Limits}

Our goal in the next two sections will be gaining an understanding of $\Prob$ from a more logical perspective. An absolute prerequisite for these discussions is a good understanding of finite limits in our topos. We will need products for controlling the number of free variables in our formulas and equalizers for expressing that two terms are equal.

The idea is to first understand finite limits in $\FinProb$, since the sheaves we are interested in are coproducts of representables. But the situation is not as simple as one may hope: finite limits in $\FinProb$ only very rarely exist. Naturally, the presheaves $\Set^{\FinProb^{op}}$ actually do have these finite limits, and we will see that they have a straightforward form. This will give rise to a weakened form of the notion of a limit, which we will call multi-limit. It will be essential in all our discussions about logic.

The most surprising part about multi-limits will be their nontrivial combinatorial content and their connection to polytopes. We will see that an many statements in this paper are proved by using computations with multi-limits and polytopes.

We introduce convenient notation to emphazise the combinatorial nature of finite probability spaces.

\begin{definition}[U-notation] \label{U-notation}
    \mbox{}
    \begin{enumerate}
        \item For $r_1, \cdots r_n \in [0,1]$ such that $r_1 + \cdots + r_n = 1$, denote by $U_{r_1 \cdots r_n}$ or simply by $U_{r_i}$ the finite probability space on the set
        \[
        \{i \mid r_i \neq 0 \}
        \]
        equipped with the measure $\mu(i) = r_i$.
        \item We will use the shorthand $U_r := U_{r,r-1}$ for $r \in [0,1]$.
    \end{enumerate}
\end{definition}

This is notation for finite probability spaces in the sense that the $U_{r_i}$ form a skeleton of $\FinProb$.

\begin{remark}\label{classifying_partitions}
    Notice that for a finite probability space $(X,\mu)$, $\Hom((X,\mu), U_{r_i})$ precisely consists of partitions $(a_i)$ of the set $X$ with $\mu(a_i) = r_i$. This may seem like quite a harmless observation, but being aware of this fact makes some calculations with $\y U_{r_i}$ much more intuitive. Moreover, we will later see that the $U_{r_i}$ in fact precisely correspond to logical formulas specifying a partition with fixed measures. We will often identify maps with partitions implicitly and without comment.

    In particular, $\Hom((X,\mu), U_r)$ is identified with certain partitions consisting of two elements, or equivalently, subsets of $X$ of measure $r$. We will identify $\Hom((X,\mu), U_r)$ with such subsets.

\end{remark}

We now compute the finite limits of representables in $\Set^{\FinProb^{op}}$ (or equivalently in $\Prob$).

\begin{proposition} \label{multi-product}
    Let $U_{r_1, \cdots r_n}$ and $U_{s_1, \cdots s_m}$ be finite probability spaces. For each finite probability space $U_{t_{ij}}$\footnote{Technically, $t_{ij}$ has two indices, so one would have to reindex it with one index so that our notation is well-defined. Subsequently, we will ignore this issue.} ($0 < i \leq n$, $0 < j \leq m$) with
    \[
        \sum_j t_{ij} = r_i
    \]
    and
    \[
        \sum_i t_{ij} = s_j ,
    \]
    we define projections $p_1: U_{t_{ij}} \to U_{r_i}$, $(x_{ij}) \mapsto (\bigvee_j x_{ij})$ and $p_2: U_{t_{ij}} \to U_{s_i}$, $(x_{ij}) \mapsto (\bigvee_i x_{ij})$.

    These projection maps induce the isomorphism
    \[
    \y U_{r_i} \times \y U_{s_j} \cong \coprod_{t_{ij}} \y U_{t_{ij}} ,
    \]
    where the indexing $t_{ij}$ range over the ones with the fixed sums specified above.
\end{proposition}
\begin{proof}
    The $t_{ij}$ where precisely chosen such that the projections are measure preserving. To show
    \[
    \y U_{r_i} \times \y U_{s_j} \cong \coprod_{t_{ij}} \y U_{t_{ij}} ,
    \]
    send a pair of partitions $((a_i),(b_j))$ to the family $(a_i \wedge b_j)_{ij}$. The inverse of this map is given by joining the family with the projection maps described above.
\end{proof}

Notice that the isomorphism we proved is not quite a universal property in the traditional sense. Normally, we would have required $\y U_{r_i} \times \y U_{s_j}$ to actually be representable in order to be able to say that we have a products in $\FinProb$. But this does not work in our case, as it only is a coproduct of representables.

We give a name to this weaker notion of a limit.

\begin{definition}
    Given a small category $\C$ and a diagram $D: J \to C$, we will call a family of cones\footnote{We think of cones over $D$ as natural transformations $c \Rightarrow D$ from a constant diagram $c$ to the diagram $D$.} $(d_i \Rightarrow D)_i$ over $D$ a \emph{multi-limit} of $D$ if the cones induce an isomorphism
    \[
        \limit_{j \in J} \y D(j) \cong \coprod_i \y d_i
    \]
    in $\Set^{\C^{op}}$.

    Moreover, we will use the terminology \emph{multi-product}, \emph{multi-equalizer} and \emph{multi-pullback} to refer to multi-limits of particular shapes.
\end{definition}

Alternatively, we could have presented the universal property in a more traditional style by saying that cones over our diagram factor through one of the $U_{t_{ij}}$ uniquely. In fact, this is what is done in \cite{caramello2019some} (Lemma 6.11.). Caramello and Lafforgue use multi-products in order to study the atoms of an atomic topos, which is also one of the things we will use them for.

Finally, we also have multi-equalizers.

\begin{proposition}
    Given the diagram \begin{tikzcd}[column sep=small]
        \alpha \arrow[r, "g", shift left] \arrow[r, "h"', shift right] & \beta
        \end{tikzcd} in $\FinProb$ we have
    \[
        eq(\begin{tikzcd}[column sep=small]
        \y \alpha \arrow[r, "g", shift left] \arrow[r, "h"', shift right] & \y \beta
        \end{tikzcd}) \cong \y \alpha \text{ if } f = g \text{ and } \emptyset \text{ otherwise.}
    \]
\end{proposition}
\begin{proof}
    The interesting case is when $f \neq g$. In that case we compute the equalizer pointwise on a finite probability space $\gamma$:
    \[
        eq(\begin{tikzcd}[column sep=small]
        \y \alpha \arrow[r, "g", shift left] \arrow[r, "h"', shift right] & \y \beta
        \end{tikzcd})(\gamma) \cong \{ h: \gamma \twoheadrightarrow \alpha \mid fh = gh \} \cong \emptyset .
    \]
\end{proof}


\subsection{A calculus of polytopes}

We have seen that treating multi-limits requires us to talk about coproducts over certain index sets. As we progress further into the paper, we will encounter many more of these constructions. In order to describe these coproducts in a concise manner, we introduce simplified notation for them.

\begin{definition}
    \mbox{}
    \begin{enumerate}
        \item Let $\mathcal{A}$ denote the set
        \[
            \{(r_1,\cdots, r_n) \mid n \in \mathbb{N}, r_i \in \mathbb{R}_{\geq 0}, \sum r_i = 1\}.
        \]
        \item Define a \emph{set of atoms} to be a pair $(S, T)$, where $S$ is a subset of $\mathcal{A}$ and $T(x)$ is a set for all $x \in S$.
        \item The sheaf associated to a set of atoms $(S,T)$ will be defined by
        \[
            \la T(x) \mid x \in S \ra := \coprod_{\substack{x \in S \\ y \in T(x)}} U_x ,
        \]
        using U-notation.
        \item An \emph{element} of a set of atoms $(S,T)$ will be a tuple $(x,t) \in \coprod_{x \in S} T(x)$. If $T(x)$ is always either empty or the one-element set, we will abbreviate and say that $x$ is an \emph{element} of $(S,T)$.
    \end{enumerate}
\end{definition}

One subtlety about our definitions is that the set $\mathcal{A}$ will contain multiple distinct elements corresponding to the same atom, simply because we are considering \emph{ordered} families $r_i$ and the space $U_{r_i}$ does not depend on the order.

We have seen that when thinking about binary multi-products, subsets $(t_{ij})$ of $\mathcal{A}$ with fixed sums of rows and columns are essential. Since multi-products will be invaluable in our study of $\Prob$, we introduce some notation for these index sets.

\begin{definition}
    Let $n_1, \cdots, n_k$ be positive integers and for each $j = 1, \cdots k$, let $r^j_1, \cdots r^j_{n_j}$ be nonnegative real numbers summing up to $1$. Now define:
    \begin{equation*}
        \begin{split}
            P^k_{r_i^1,\cdots r_i^k} := \{ (t_{i_1 \cdots i_k}) &\in \mathbb{R}^{n_1 \times \cdots \times n_k} \mid t_{i_1 \cdots i_k} \geq 0, \\
            & \sum_{j=1}^{n_1} t_{j i_2 \cdots i_k} = r_1, \\
            & \cdots \\
            & \sum_{j=1}^{n_k} t_{i_1 i_2 \cdots j} = r_k \\
            & \text{ for all indices } i_1, \cdots i_k \}
        \end{split}
    \end{equation*}
\end{definition}

\begin{example} \label{notation_example}
    \mbox{}
    \begin{itemize}
        \item Binary multi-products can now be written as:
        \[ \y U_{r_i} \times \y U_{s_j} \cong \la 1 \mid t_{ij} \in P^2_{r_i,s_j} \ra, \]
        where $1$ denotes the one-object set.
        \item Since $\FinProb$ has a terminal object, we can use our binary multi-product formula in order to compute arbitrary finite multi-products:
        \[
            \prod_{j=1}^n \y U_{r_i^j} \cong \la 1 \mid t_{i^1 \cdots i^n} \in P^n_{r_i^1 \cdots r_i^n} \ra
        \]
        \item Computing the multi-pullback of a diagram
        \[
            \begin{tikzcd}
            & U_{r'_i} \arrow[d, "g"] \\
            U_{r_i} \arrow[r, "f"'] & U_{s_i} ,
            \end{tikzcd}
        \]
        (by noting that equalizers of coproducts are coproducts of equalizers in presheaf topoi, which can be checked pointwise) yields the formula
        \begin{equation*}
            \begin{split}
            \y U_{r_i} &\times_{\y U_{s_i}} \y U_{r_i'} \cong \\
            &\la 1 \mid t_{ij} \in P^2_{r_i, r_i'}, t_{ij} = 0 \text{ for all } i,j \text{ with } f(i) \neq g(j) \ra
            \end{split}
        \end{equation*}
        \item Define the presheaf $U$ in the following way:
        \[ U \cong \la 1 \mid (r_1,r_2) \in P^1_2 \ra. \]
        We will later see that this presheaf is in fact the underlying sheaf of a universal model in $\Prob$ for a suitable theory of boolean algebras with measures.
    \end{itemize}
\end{example}

\begin{remark} We have seen that the elements $t_{ij}$ of a binary multi-product are the $t_{ij} \in P^2_{r_i,s_i}$. This set can be seen a subset of the set of $n\times m$-matrices, which is, in fact, a polytope in $\mathbb{R}^{n\times m}$. The combinatorics literature has called these polytopes \emph{transportation polytopes}. Chapter 8 of [Matrix book] is dedicated to their study.
\end{remark}

One should note that when specifying a sheaf $\mathcal{F}$ by using this set of atoms notation, the elements of $\mathcal{F}$ (as defined above) are precisely the monomorphisms $\y U_{r_i} \hookrightarrow \mathcal{F}$.

Unfortunately, one can't specify all the sheaves in this way. Nevertheless, this notation can be easily adapted to incorporate the non-representable atoms we discuss in \ref{atoms}, but we won't need this in this article.

The most convenient thing about sets of atoms, is that it is easy to send them through (the inverse image functor of) a geometric morphism, since geometric morphisms preserve coproducts. In fact, we found that the only practical way to compute geometric morphisms on a sheaf, is by decomposing the sheaf into atoms. This will be done numerous times throughout the article. In fact, we will occasionally also need to send morphisms in $\Prob$ through inverse image functors, so we also describe a mechanism through which to decompose morphisms:

\begin{proposition}\label{atom_coprod_maps}
    Let $\mathcal{F} = \la T(x) \mid x \in S \ra$ and $\mathcal{G} = \la T'(x) \mid x \in S' \ra $. Then the maps $\mathcal{F} \to \mathcal{G}$ are precisely pairs $(g, (f_x))$, where $g$ is a map between the sets of elements
    \[
        \coprod_{x \in S} T(x) \to \coprod_{x \in S'} T'(x)
    \]
    and for each element $(x,t) \in \coprod_{x \in S} T(x)$ with $(x',t') = g(x,t)$, $f_{x,t}$ is a morphism
    \[
        U_x \xrightarrow{f_{x,t}} U_{x'} .
    \]
\end{proposition}
\begin{proof}
    Use the universal property of the coproduct of atoms and then use that each map needs to factor through one of the images because the image of an atom needs to be an atom again.
\end{proof}

\subsection{A better sheaf condition}

As a toy application of the facts discovered so far, we develop a better sheaf condition. Since we won't use it anywhere, the reader can safely skip this subsection.

\begin{proposition} \label{sheaf_condition}
A presheaf $\mathcal{F}$ on $\FinProb$ is a sheaf over the atomic topology if and only if for all maps of finite probability spaces $f: U_{r_i} \to U_{s_i}$ the fork
\[
\begin{tikzcd}
\mathcal{F} U_{s_i} \arrow[r, "\mathcal{F}f"] &
\mathcal{F} U_{r_i} \arrow[r, "\pi_1", shift left] \arrow[r, "\pi_2"', shift right] & \prod_{t_{ij}} \mathcal{F} U_{t_{ij}}
\end{tikzcd}
\]
is an equalizer diagram, where the $t_{ij}$ range over the elements of the multi-pullback $U_{r_i} \times_{U_{s_i}} U_{r_i}$ (we are pulling $f$ back along itself) and the projection $\pi_1$ and $\pi_2$ are induced by the universal 'projection maps' of multi-pullbacks.
\end{proposition}
\begin{proof}
Since for each $U_{t_{ij}}$ we have the commuting square
\[
\begin{tikzcd}
U_{t_{ij}} \arrow[r] \arrow[d] & {U_{r_i}} \arrow[d, "f"] \\
{U_{r_i}} \arrow[r, "f"']            & {U_{s_i}} ,
\end{tikzcd}
\]
$f$ certainly gives us a diagram
\[
\begin{tikzcd}
\mathcal{F} U_{s_i} \arrow[r, "\mathcal{F}f"] & \mathcal{F} U_{r_i} \arrow[r, "\pi_1", shift left] \arrow[r, "\pi_2"', shift right] & \prod_{t_{ij}} \mathcal{F} U_{t_{ij}}
\end{tikzcd}
\]
with $\pi_1 \mathcal{F}(f) = \pi_2 \mathcal{F}(f)$. Claiming that this is an equalizer means saying that for every $a \in \mathcal{F}U_{r_i}$ with the property that for all the projections $p_1: U_{t_{ij}} \to U_{r_i}$, $p_2: U_{t_{ij}} \to U_{r'_i}$ the equation $\mathcal{F}(p_1)(a) = \mathcal{F}(p_2)(a)$ is satisfied, implies that there is a unique lift $b \in \mathcal{F}U_{s_i}$ of $a$.

So our claim is that the criterion described above needs only be checked on the $U_{t_{ij}}$. Thus we assume that the equation above holds for the $U_{t_{ij}}$ with their projection maps. We need to show that the equation $\mathcal{F}(g)(a) = \mathcal{F}(h)(a)$ already holds for all the \[
\begin{tikzcd}
\gamma \arrow[r, "g", shift left] \arrow[r, "h"', shift right] & U_{r_i}
\end{tikzcd}
\]
with $fg = fh$. But this certainly holds, since we can factor $g$ and $h$ through one of the $U_{t_{ij}}$ by the universal property of the multi-pullback.
\end{proof}

Note that this proposition actually had little to do with finite probability spaces and is really about a sheaf condition for atomic topologies where the base category has multi-pullbacks.


\bibliographystyle{unsrt}
\bibliography{probability-topos}

\end{document}

\documentclass[a4paper]{amsproc}
\renewcommand\labelenumi{(\roman{enumi})}
\renewcommand\theenumi\labelenumi
\title{An Atomic Topos for Probability Theory}

\usepackage[english]{babel}
\usepackage{amsmath}
\usepackage{amssymb}
\usepackage{amsthm}
\usepackage{fontspec}
\usepackage[hcentering,bindingoffset=0mm]{geometry}
\usepackage{tikz}
\usepackage{lscape}
\usepackage{mathtools}
% for clickable references
\usepackage{hyperref}\urlstyle{rm}
\usetikzlibrary{cd}

% I like how the sans serif doesn't stand out as much in this font.
% You need to have the `gnu-free-fonts` package installed (on arch)
\setmainfont{FreeSerif}

\theoremstyle{plain}
 \newtheorem{theorem}{Theorem}[section]
 \newtheorem{proposition}[theorem]{Proposition}
 \newtheorem{lemma}[theorem]{Lemma}
 \newtheorem{corollary}[theorem]{Corollary}
\theoremstyle{definition}
 \newtheorem{example}[theorem]{Example}
 \newtheorem{definition}[theorem]{Definition}
 \newtheorem{notation}[theorem]{Notation}
\theoremstyle{remark}
 \newtheorem{remark}[theorem]{Remark}
 \numberwithin{equation}{section}

% Some shortcuts
% \newcommand{\ob}[1]{ob(#1)}
\newcommand{\id}{\textup{id}}

\DeclareMathOperator{\Hom}{Hom}
\DeclareMathOperator{\Sh}{Sh}
\newcommand{\y}{\textit{y}}
% The star indicates that subscripts are to be typeset below the operator
% instead of to the right
\DeclareMathOperator*{\limit}{lim}
\DeclareMathOperator*{\colim}{colim}
\DeclareMathOperator{\Lan}{Lan}
\DeclareMathOperator{\eq}{eq}
\newcommand{\s}{\textsf{ }}

\newcommand{\Set}{\textup{\textsf{Set}}}
\newcommand{\FinProb}{\textup{\textsf{FinProb}}}
\newcommand{\BPAlg}{\textup{\textsf{BPAlg}}}
\newcommand{\Ind}{\textsf{Ind}}
\newcommand{\C}{\mathcal{C}}
\newcommand{\Prob}{\mathfrak{Prob}}

\newcommand{\la}{\langle\,}
\newcommand{\ra}{\,\rangle}

\newcommand{\ldoub}{[\![}
\newcommand{\rdoub}{]\!]}
\makeatletter
\newcommand{\rmnum}[1]{\romannumeral #1}
\newcommand{\Rmnum}[1]{\expandafter\@slowromancap\romannumeral #1@}
\makeatother

\author{Adam Dauser, Adrian Marti}

\date{}
\begin{document}

\begin{abstract}
WIP
\end{abstract}

\maketitle

\tableofcontents


\section{Introduction}

WIP

\section{A combinatorial construction of $\Prob$}

In this section, we define the topos $\Prob$ as a category of sheaves on the category of finite probability spaces. We then proceed by proving properties about this construction, which we will need throughout the paper.


\begin{definition}
    \mbox{}
    \begin{enumerate}
        \item A finite probability space is a finite set $X$ equipped with a map (called probability measure) $\mu: X \to \mathbb{R}_{>0}$ such that \\ $\sum_{x \in X} \mu(x) = 1$.
        \item For a finite probability space $(X,\mu)$ and a subset $a \subset X$, we define $\mu(a) := \sum_{x \in a} \mu(x)$.
        \item A map of finite probability spaces $(X,\mu) \to (Y,\nu)$ is defined as a map of finite sets $f: X \to Y$ that is measure-preserving, i.e. $\mu(f^{-1}(y)) = \nu(y)$. Denote the category of finite probability spaces by $\FinProb$.
    \end{enumerate}
\end{definition}

Notice that all maps of finite probability spaces are surjective. Also notice that we do not allow elements of measure zero.

We want to consider the sheaf topos $\Sh(\FinProb, J_{at})$, where $J_{at}$ denotes the atomic topology. Recall that the atomic topology is defined by letting every nonempty sieve be a covering sieve. We need to check that this yields a Grothendieck topology.

\begin{proposition} \label{pullback_measure}
    Given a diagram of finite probability spaces
    \[
    \begin{tikzcd}
                                & {(X_2,\mu_2)} \arrow[d, "g"] \\
    {(X_1, \mu_1)} \arrow[r, "f"'] & {(Y,\mu)}
    \end{tikzcd},
    \]
    the pullback of the finite sets carries a probability measure $\nu(x,y) := \frac{\mu_1(x) \mu_2(y)}{\mu(f x)}$ such that the diagram
    \[
    \begin{tikzcd}
    {(X_1 \times_Y X_2, \nu)} \arrow[d, "p_1"'] \arrow[r, "p_2"] & {(X_2,\mu_2)} \arrow[d, "g"] \\
    {(X_1, \mu_1)} \arrow[r, "f"']                               & {(Y,\mu)}
    \end{tikzcd}
    \]
    commutes.

    In particular, $J_{at}$ is a Grothendieck topology on $\FinProb$.
\end{proposition}
\begin{proof}
    The only thing to show is that the projections are measure-preserving, since that also implies that $\nu$ is actually a measure. So let $x \in X_1$ and $u := f(x)$.
    \begin{align*}
        \sum_{(x,y) \in p_1^{-1} x} \nu(x,y) &= \sum_{y \in g^{-1} u} \frac{\mu_1(x)\mu_2(y)}{\mu(u)} \\
        &= \frac{\mu_1(x)}{\mu(u)} \sum_{y \in g^{-1} u} \mu_2(y) \\
        &= \frac{\mu_2(x)}{\mu(u)} \mu(u) \\
        &= \mu_2(x)
    \end{align*}
    Similarly, $p_2$ is measure-preserving.

    For the second part, notice that we just proved the \emph{right Ore condition} for $\FinProb$. Checking the Grothendieck topology axioms is a known straightforward formal consequence of this.
\end{proof}

We now introduce the protagonist of this paper.

\begin{definition}
    Define the topos $\Prob$ to be $\Sh(\FinProb, J_{at})$.
\end{definition}

Much of this article is dedicated to finding other equivalent ways of describing $\Prob$. When we wish to leave the particular description of $\Prob$ vague, we will use the notation $\Prob$.

Lemma 2 in section \Rmnum{3}.4 of \cite{maclane2012sheaves} gives us a criterion for checking whether a presheaf is actually a sheaf over the atomic topology:

A presheaf $\mathcal{F}$ on $\FinProb$ is a sheaf over the atomic topology if and only if one can check the following for any morphism $f: \alpha \to \beta$ and any $a \in \mathcal{F}(\alpha)$: If for all diagrams

\[
\begin{tikzcd}
\gamma \arrow[r, "g", shift left] \arrow[r, "h"', shift right] & \alpha \arrow[r, "f"] & \beta
\end{tikzcd}
\]
with $f g = f h$, we have that $\mathcal{F}(g)(a) = \mathcal{F}(h)(a)$, then there is a unique lift $b \in \mathcal{F}(\beta)$ with the property that $\mathcal{F}(f)(b) = a$.

In some cases, one can prove the following condition that is much stronger than being a sheaf. It is especially useful when the sheaf doesn't really use the measure in its definition.

\begin{proposition} \label{sheaf_condition_weak}
Let $\mathcal{F}$ be a presheaf on $\FinProb$. If for all morphisms $f: (X,\mu) \to (Y,\nu)$ the fork
\[
\begin{tikzcd} \mathcal{F} (Y,\nu) \arrow[r, "\mathcal{F}f"] & \mathcal{F} (X,\mu) \arrow[r, "\pi_1", shift left] \arrow[r, "\pi_2"', shift right] & \mathcal{F} (X \times_f X,\bar{\mu})
\end{tikzcd}
\]
is an equalizer, where $\bar{\mu}$ is as in \ref{pullback_measure}, then $\mathcal{F}$ is a sheaf.
\end{proposition}
\begin{proof}
This is the condition from \cite{maclane2012sheaves} described above, but weakened by choosing $\gamma = (X \times_f X,\bar{\mu})$.
\end{proof}


We can now check the sheaf condition for the vast majority of sheaves that we are interested in:

\begin{proposition}\label{subcanonical}
    \mbox{}
    \begin{enumerate}
        \item All morphisms in $\FinProb$ are regular epimorphisms.
        \item The representable functors are sheaves, i.e. $J_{at}$ is a subcanonical topology on $\FinProb$.
        \item All coproducts of representable functors are sheaves.
    \end{enumerate}
\end{proposition}
\begin{proof}
    \mbox{}
    \begin{enumerate}
        \item We need to show that every morphism $f:(X,\mu)\rightarrow (Y, \nu)$ comes from a coequalizer. Our construction of the measure on the pullback in \ref{pullback_measure} gives us the following fork:
        \[
        \begin{tikzcd}[column sep=large]
        (X \times_Y X, \bar{\mu}) \arrow[r, shift left=2] \arrow[r, shift right] & (X,\mu) \arrow[r, "f", two heads] & (Y,\nu)
        \end{tikzcd}
        \]
        Since this is clearly a coequalizer on the underlying sets, the only thing to check is that the uniquely induced maps are measure-preserving. This is straightforward to verify.

        \item Checking that the representable functors are sheaves by using \ref{sheaf_condition_weak} reduces to the fact that the fork above is a coequalizer diagram.

        \item We check the sheaf condition \ref{sheaf_condition_weak} for a coproduct of representables. But since we already know this sheaf condition holds for representables, we merely need to note that equalizers of coproducts are coproducts of equalizers in presheaf topoi (since this fact can be checked pointwise in set).
    \end{enumerate}
\end{proof}

Note that the second and third part are merely formal consequences of the first one. Any category (satisfying the \emph{right Ore condition}) such that the first part holds, also satisfies the second and the third part.

Recall that a site $(\C, J)$, is called \emph{atomic} if every nonempty sieve over an object $c$ is covering. Furthermore, a Grothendieck topos is called \emph{atomic} if it is equivalent to a topos of sheaves over an atomic site. Tautologically, $\Prob$ can be seen to be an atomic topos. An object in an atomic topos is called an \emph{atom} if it has precisely two subobjects (the initial object and the object itself).

We just saw that the representables $\y U_{r_i}$ are actually sheaves($\y$ denotes the Yoneda embedding). In fact, since $\Prob$ is an atomic topos, these representables are atoms. Moreover, for atomic topoi, every object can be uniquely written as a coproduct of atoms. So in some sense, we can reduce all problems to atoms. The natural question arises: Are there more atoms than the representables? In \ref{atoms} we will give a complete list of the atoms and see that the representables are not all of them.


\subsection{Finite Limits}

Our goal in the next two sections will be gaining an understanding of $\Prob$ from a more logical perspective. An absolute prerequisite for these discussions is a good understanding of finite limits in our topos. We will need products for controlling the number of free variables in our formulas and equalizers for expressing that two terms are equal.

The idea is to first understand finite limits in $\FinProb$, since the sheaves we are interested in are coproducts of representables. But the situation is not as simple as one may hope: finite limits in $\FinProb$ only very rarely exist. Naturally, the presheaves $\Set^{\FinProb^{op}}$ actually do have these finite limits, and we will see that they have a straightforward form. This will give rise to a weakened form of the notion of a limit, which we will call multi-limit. It will be essential in all our discussions about logic.

The most surprising part about multi-limits will be their nontrivial combinatorial content and their connection to polytopes. We will see that an many statements in this paper are proved by using computations with multi-limits and polytopes.

We introduce convenient notation to emphazise the combinatorial nature of finite probability spaces.

\begin{definition}[U-notation] \label{U-notation}
    \mbox{}
    \begin{enumerate}
        \item For $r_1, \cdots r_n \in [0,1]$ such that $r_1 + \cdots + r_n = 1$, denote by $U_{r_1 \cdots r_n}$ or simply by $U_{r_i}$ the finite probability space on the set
        \[
        \{i \mid r_i \neq 0 \}
        \]
        equipped with the measure $\mu(i) = r_i$.
        \item We will use the shorthand $U_r := U_{r,r-1}$ for $r \in [0,1]$.
    \end{enumerate}
\end{definition}

This is notation for finite probability spaces in the sense that the $U_{r_i}$ form a skeleton of $\FinProb$.

\begin{remark}\label{classifying_partitions}
    Notice that for a finite probability space $(X,\mu)$, $\Hom((X,\mu), U_{r_i})$ precisely consists of partitions $(a_i)$ of the set $X$ with $\mu(a_i) = r_i$. This may seem like quite a harmless observation, but being aware of this fact makes some calculations with $\y U_{r_i}$ much more intuitive. Moreover, we will later see that the $U_{r_i}$ in fact precisely correspond to logical formulas specifying a partition with fixed measures. We will often identify maps with partitions implicitly and without comment.

    In particular, $\Hom((X,\mu), U_r)$ is identified with certain partitions consisting of two elements, or equivalently, subsets of $X$ of measure $r$. We will identify $\Hom((X,\mu), U_r)$ with such subsets.

\end{remark}

We now compute the finite limits of representables in $\Set^{\FinProb^{op}}$ (or equivalently in $\Prob$).

\begin{proposition} \label{multi-product}
    Let $U_{r_1, \cdots r_n}$ and $U_{s_1, \cdots s_m}$ be finite probability spaces. For each finite probability space $U_{t_{ij}}$\footnote{Technically, $t_{ij}$ has two indices, so one would have to reindex it with one index so that our notation is well-defined. Subsequently, we will ignore this issue.} ($0 < i \leq n$, $0 < j \leq m$) with
    \[
        \sum_j t_{ij} = r_i
    \]
    and
    \[
        \sum_i t_{ij} = s_j ,
    \]
    we define projections $p_1: U_{t_{ij}} \to U_{r_i}$, $(x_{ij}) \mapsto (\bigvee_j x_{ij})$ and $p_2: U_{t_{ij}} \to U_{s_i}$, $(x_{ij}) \mapsto (\bigvee_i x_{ij})$.

    These projection maps induce the isomorphism
    \[
    \y U_{r_i} \times \y U_{s_j} \cong \coprod_{t_{ij}} \y U_{t_{ij}} ,
    \]
    where the indexing $t_{ij}$ range over the ones with the fixed sums specified above.
\end{proposition}
\begin{proof}
    The $t_{ij}$ where precisely chosen such that the projections are measure preserving. To show
    \[
    \y U_{r_i} \times \y U_{s_j} \cong \coprod_{t_{ij}} \y U_{t_{ij}} ,
    \]
    send a pair of partitions $((a_i),(b_j))$ to the family $(a_i \wedge b_j)_{ij}$. The inverse of this map is given by joining the family with the projection maps described above.
\end{proof}

Notice that the isomorphism we proved is not quite a universal property in the traditional sense. Normally, we would have required $\y U_{r_i} \times \y U_{s_j}$ to actually be representable in order to be able to say that we have a products in $\FinProb$. But this does not work in our case, as it only is a coproduct of representables.

We give a name to this weaker notion of a limit.

\begin{definition}
    Given a small category $\C$ and a diagram $D: J \to C$, we will call a family of cones\footnote{We think of cones over $D$ as natural transformations $c \Rightarrow D$ from a constant diagram $c$ to the diagram $D$.} $(d_i \Rightarrow D)_i$ over $D$ a \emph{multi-limit} of $D$ if the cones induce an isomorphism
    \[
        \limit_{j \in J} \y D(j) \cong \coprod_i \y d_i
    \]
    in $\Set^{\C^{op}}$.

    Moreover, we will use the terminology \emph{multi-product}, \emph{multi-equalizer} and \emph{multi-pullback} to refer to multi-limits of particular shapes.
\end{definition}

Alternatively, we could have presented the universal property in a more traditional style by saying that cones over our diagram factor through one of the $U_{t_{ij}}$ uniquely. In fact, this is what is done in \cite{caramello2019some} (Lemma 6.11.). Caramello and Lafforgue use multi-products in order to study the atoms of an atomic topos, which is also one of the things we will use them for.

Finally, we also have multi-equalizers.

\begin{proposition}
    Given the diagram \begin{tikzcd}[column sep=small]
        \alpha \arrow[r, "g", shift left] \arrow[r, "h"', shift right] & \beta
        \end{tikzcd} in $\FinProb$ we have
    \[
        eq(\begin{tikzcd}[column sep=small]
        \y \alpha \arrow[r, "g", shift left] \arrow[r, "h"', shift right] & \y \beta
        \end{tikzcd}) \cong \y \alpha \text{ if } f = g \text{ and } \emptyset \text{ otherwise.}
    \]
\end{proposition}
\begin{proof}
    The interesting case is when $f \neq g$. In that case we compute the equalizer pointwise on a finite probability space $\gamma$:
    \[
        eq(\begin{tikzcd}[column sep=small]
        \y \alpha \arrow[r, "g", shift left] \arrow[r, "h"', shift right] & \y \beta
        \end{tikzcd})(\gamma) \cong \{ h: \gamma \twoheadrightarrow \alpha \mid fh = gh \} \cong \emptyset .
    \]
\end{proof}


\subsection{A calculus of polytopes}

We have seen that treating multi-limits requires us to talk about coproducts over certain index sets. As we progress further into the paper, we will encounter many more of these constructions. In order to describe these coproducts in a concise manner, we introduce simplified notation for them.

\begin{definition}
    \mbox{}
    \begin{enumerate}
        \item Let $\mathcal{A}$ denote the set
        \[
            \{(r_1,\cdots, r_n) \mid n \in \mathbb{N}, r_i \in \mathbb{R}_{\geq 0}, \sum r_i = 1\}.
        \]
        \item Define a \emph{set of atoms} to be a pair $(S, T)$, where $S$ is a subset of $\mathcal{A}$ and $T(x)$ is a set for all $x \in S$.
        \item The sheaf associated to a set of atoms $(S,T)$ will be defined by
        \[
            \la T(x) \mid x \in S \ra := \coprod_{\substack{x \in S \\ y \in T(x)}} U_x ,
        \]
        using U-notation.
        \item An \emph{element} of a set of atoms $(S,T)$ will be a tuple $(x,t) \in \coprod_{x \in S} T(x)$. If $T(x)$ is always either empty or the one-element set, we will abbreviate writing $(x,t)$ for an element of $(S,T)$ by merely stating that $x$ is an element of $(S,T)$.
    \end{enumerate}
\end{definition}

One subtlety about our definitions is that the set $\mathcal{A}$ will contain multiple distinct elements corresponding to the same atom, simply because we are considering \emph{ordered} families $r_i$ and the space $U_{r_i}$ does not depend on the order.

Notice that the elements of a set of atoms $(S,T)$ index the coproduct in the defintion of $\la T(x) \mid x \in S \ra$. Moreover they can also be seen as the monomorphisms
\[
    \y U_{r_i} \hookrightarrow \la T(x) \mid x \in S \ra
\]
into the sheaf associated to $(S,T)$.

Occasionally, we may add an additional properties $\phi_1, \phi_2, \cdots, \phi_n$ to the notation, as follows:
\[
    \la T(x) \mid x \in S, \phi_1(x), \cdots, \phi_n(x) \ra := \la T(x) \mid x \in \{x' \in S \mid \phi_1(x'), \cdots, \phi_n(x')\} \ra
\]
While this is somewhat informal(\emph{What is a property?}), the reader most likely knows what is meant.

We have seen that when thinking about binary multi-products, subsets $(t_{ij})$ of $\mathcal{A}$ with fixed sums of rows and columns are essential. Since multi-products will be invaluable in our study of $\Prob$, we introduce some notation for these index sets.

\begin{definition}
    Let $n_1, \cdots, n_k$ be positive integers.
    \begin{enumerate}
        \item Define:
        \begin{equation*}
            P^k_{n_1,\cdots,n_k} := \{ (t_{i_1 \cdots i_k}) \in \mathbb{R}^{n_1 \times \cdots \times n_k} \mid t_{i_1 \cdots i_k} \geq 0, \sum_{i_1,\cdots,i_k} t_{i_1 \cdots i_k} = 1 \}
        \end{equation*}
        \item For each $j = 1, \cdots k$, let $r^j_1, \cdots r^j_{n_j}$ be nonnegative real numbers summing up to $1$. Now let
        \begin{equation*}
            \begin{split}
                P^k_{r_i^1,\cdots r_i^k} := \{ (t_{i_1 \cdots i_k}) &\in \mathbb{R}^{n_1 \times \cdots \times n_k} \mid t_{i_1 \cdots i_k} \geq 0, \\
                & \sum_{j=1}^{n_1} t_{j i_2 \cdots i_k} = r_1, \\
                & \cdots \\
                & \sum_{j=1}^{n_k} t_{i_1 i_2 \cdots j} = r_k \\
                & \text{ for all indices } i_1, \cdots i_k \} .
            \end{split}
        \end{equation*}
    \end{enumerate}
\end{definition}

\begin{example} \label{notation_example}
    \mbox{}
    \begin{itemize}
        \item Binary multi-products can now be written as
        \[ \y U_{r_i} \times \y U_{s_j} \cong \la 1 \mid t_{ij} \in P^2_{r_i,s_j} \ra, \]
        where $1$ denotes the one-object set.
        \item Since $\FinProb$ has a terminal object, we can use our binary multi-product formula in order to compute arbitrary finite multi-products:
        \[
            \prod_{j=1}^n \y U_{r_i^j} \cong \la 1 \mid t_{i^1 \cdots i^n} \in P^n_{r_i^1 \cdots r_i^n} \ra
        \]
        \item Computing the multi-pullback of a diagram
        \[
            \begin{tikzcd}
            & U_{r'_i} \arrow[d, "g"] \\
            U_{r_i} \arrow[r, "f"'] & U_{s_i} ,
            \end{tikzcd}
        \]
        (by noting that equalizers of coproducts are coproducts of equalizers in presheaf topoi, which can be checked pointwise), yields the formula
        \begin{equation*}
            \begin{split}
            \y U_{r_i} &\times_{\y U_{s_i}} \y U_{r_i'} \cong \\
            &\la 1 \mid t_{ij} \in P^2_{r_i, r_i'}, t_{ij} = 0 \text{ for all } i,j \text{ with } f(i) \neq g(j) \ra .
            \end{split}
        \end{equation*}
        \item Define the presheaf $U$ in the following way:
        \[ U \cong \la 1 \mid (r_1,r_2) \in P^1_2 \ra \]
        We will later see that this presheaf is in fact the underlying sheaf of a universal model in $\Prob$ for a suitable theory of boolean algebras with measures.
    \end{itemize}
\end{example}

\begin{remark} We have seen that the elements $t_{ij}$ of a binary multi-product\footnote{We have not specified what exactly the data $(S,T)$ is, for which we want to consider elements. In this case, and in the future, we trust that the reader is able to infer this data on his own.} are the $t_{ij} \in P^2_{r_i,s_i}$. This set can be seen a subset of the set of $n\times m$-matrices, which is, in fact, a polytope in $\mathbb{R}^{n\times m}$. The combinatorics literature has called these polytopes \emph{transportation polytopes}. Chapter 8 of \cite{brualdi2006combinatorial} is dedicated to their study.
\end{remark}

Notice that all the sheaves we have worked with until now can be written as sheaves associated to sets of atoms. One might hope that every sheaf can be written in this form, but as remarked earlier, this is not possible, since there are non-representable atoms. Nevertheless, our definitions can be easily extended to incorporate the non-representable atoms we discuss in \ref{atoms}, but we won't need this in this article.

The most convenient thing about sets of atoms, is that it is easy to send them through (the inverse image functor of) a geometric morphism, since geometric morphisms preserve coproducts. In fact, we found that the only practical way to compute geometric morphisms on a sheaf is by decomposing the sheaf into atoms. This will be done numerous times throughout the article. In fact, we will occasionally also need to send morphisms in $\Prob$ through inverse image functors, so we also describe a mechanism through which to decompose morphisms:

\begin{proposition}\label{atom_coprod_maps}
    Let $(S,T)$ and $(S',T')$ be sets of atoms and let $\mathcal{F}$ and $\mathcal{G}$ be their associated sheaves, respectively. Then the maps $\mathcal{F} \to \mathcal{G}$ are precisely pairs $(g, (f_x))$, where $g$ is a map between the sets of elements
    \[
        \coprod_{x \in S} T(x) \to \coprod_{x \in S'} T'(x)
    \]
    and for each element $(x,t) \in \coprod_{x \in S} T(x)$ with $(x',t') = g(x,t)$, $f_{x,t}$ is a morphism
    \[
        U_x \xrightarrow{f_{x,t}} U_{x'} .
    \]
\end{proposition}
\begin{proof}
    Use the universal property of the coproduct of atoms and then use that each map needs to factor through one of the images because the image of an atom needs to be an atom again.
\end{proof}

\subsection{A better sheaf condition}

As a toy application of the facts discovered so far, we develop a better sheaf condition. Since we won't use it anywhere, the reader can safely skip this subsection.

\begin{proposition} \label{sheaf_condition}
A presheaf $\mathcal{F}$ on $\FinProb$ is a sheaf over the atomic topology if and only if for all maps of finite probability spaces $f: U_{r_i} \to U_{s_i}$ the fork
\[
\begin{tikzcd}
\mathcal{F} U_{s_i} \arrow[r, "\mathcal{F}f"] &
\mathcal{F} U_{r_i} \arrow[r, "\pi_1", shift left] \arrow[r, "\pi_2"', shift right] & \prod_{t_{ij}} \mathcal{F} U_{t_{ij}}
\end{tikzcd}
\]
is an equalizer diagram, where the $t_{ij}$ range over the elements of the multi-pullback $U_{r_i} \times_{U_{s_i}} U_{r_i}$ (we are pulling $f$ back along itself) and the projection $\pi_1$ and $\pi_2$ are induced by the universal 'projection maps' of multi-pullbacks.
\end{proposition}
\begin{proof}
Since for each $U_{t_{ij}}$ we have the commuting square
\[
\begin{tikzcd}
U_{t_{ij}} \arrow[r] \arrow[d] & {U_{r_i}} \arrow[d, "f"] \\
{U_{r_i}} \arrow[r, "f"']            & {U_{s_i}} ,
\end{tikzcd}
\]
$f$ certainly gives us a diagram
\[
\begin{tikzcd}
\mathcal{F} U_{s_i} \arrow[r, "\mathcal{F}f"] & \mathcal{F} U_{r_i} \arrow[r, "\pi_1", shift left] \arrow[r, "\pi_2"', shift right] & \prod_{t_{ij}} \mathcal{F} U_{t_{ij}}
\end{tikzcd}
\]
with $\pi_1 \mathcal{F}(f) = \pi_2 \mathcal{F}(f)$. Claiming that this is an equalizer means saying that for every $a \in \mathcal{F}U_{r_i}$ with the property that for all the projections $p_1: U_{t_{ij}} \to U_{r_i}$, $p_2: U_{t_{ij}} \to U_{r'_i}$ the equation $\mathcal{F}(p_1)(a) = \mathcal{F}(p_2)(a)$ is satisfied, implies that there is a unique lift $b \in \mathcal{F}U_{s_i}$ of $a$.

So our claim is that the criterion described above needs only be checked on the $U_{t_{ij}}$. Thus we assume that the equation above holds for the $U_{t_{ij}}$ with their projection maps. We need to show that the equation $\mathcal{F}(g)(a) = \mathcal{F}(h)(a)$ already holds for all the \[
\begin{tikzcd}
\gamma \arrow[r, "g", shift left] \arrow[r, "h"', shift right] & U_{r_i}
\end{tikzcd}
\]
with $fg = fh$. But this certainly holds, since we can factor $g$ and $h$ through one of the $U_{t_{ij}}$ by the universal property of the multi-pullback.
\end{proof}

Note that this proposition actually had little to do with finite probability spaces and is really about a sheaf condition for atomic topologies where the base category has multi-pullbacks.


\section{$\Prob$ as the classifying topos of a theory of intervals}

In this section we will show that $\Sh(\FinProb, J_{at})$ classifies a certain theory of intervals. For the most part, we will use the notations and definitions as in part D.1 of \cite{johnstone2002sketches2}. We warn the reader that this section contains big amounts of technical property-checking, so the uninterested reader can read the next definition and then skip to \ref{theory_of_intevals}.

\begin{definition}
    Define the geometric theory $\mathbb{T}_{bpalg}$ of boolean probability algebras to have one sort $B$, constants and function symbols
    \begin{center}
        $1: B$ \\
        $0: B$ \\
        $\wedge: B \times B \to B$ \\
        $\vee: B \times B \to B$ \\
        $\neg: B \to B$
    \end{center}
    and a unary relation symbol $B_r$ on $B$ for each $r \in [0,1]$. We impose the following axioms($a$, $b$ and $c$ denote free variables in $B$):
    \begin{itemize}
        \item \textit{Boolean algebra}.
        \begin{equation*}
            \begin{split}
                \top &\vdash a \wedge (b \wedge c) = (a \wedge b) \wedge c \\
                \top &\vdash a \wedge 1 = a \\
                \top &\vdash a \wedge \neg{a} = 0 \\
                \top &\vdash a \wedge (a \vee b) = a \\
                \top &\vdash a \wedge (b \vee c) = (a \wedge b) \vee (a \wedge c) \\
            \end{split}
            \quad
            \begin{split}
                \top &\vdash a \vee (b \vee c) = (a \vee b) \vee c \\
                \top &\vdash a \vee 0 = a \\
                \top &\vdash a \vee \neg{a} = 1 \\
                \top &\vdash a \vee (a \wedge b) = a \\
                \top &\vdash a \vee (b \wedge c) = (a \vee b) \wedge (a \vee c) \\
            \end{split}
            \quad
            \begin{split}
                &\textit{associativity} \\
                &\textit{identity} \\
                &\textit{inverse} \\
                &\textit{absorbtion} \\
                &\textit{distributivity} \\
            \end{split}
        \end{equation*}
        \item \textit{The $B_r$ form a partition}. For all $r, s \in [0,1]$ with $r \neq s$ we have an axiom
        \[
        B_r(a)  \wedge B_s(a) \vdash \bot
        \]
        and we also have an axiom
        \[
        \top \vdash \bigvee_{r \in [0,1]} B_r(a).
        \]
        \item \textit{Probability measure}. For all $r, s \in [0,1]$, we further require
        \[
        (a \wedge b = 0) \wedge B_r(a) \wedge B_s(b) \vdash B_{r+s}(a \vee b)
        \]
        and finally we need
        \begin{align*}
            \top & \vdash B_1(1) \\
            B_0(a) & \vdash a = 0.
        \end{align*}
    \end{itemize}
\end{definition}

Essentially, we have encoded a boolean probability algebra geometrically by partitioning the base sort $B$ into the elements $B_r$ of measure $r \in [0,1]$. Notice that we require there to only be one element of measure zero. This requirement may be somewhat surprising, since the boolean algebra of measurable sets of a $\sigma$-algebra certainly can have sets of measure zero, or \emph{nullsets}. Instead, the reader should have the boolean algebra of measurable sets \emph{modulo nullsets} in mind, as the intended model.

\begin{example}\label{mod_nullset}
\end{example}

Let us get some more notation out of the way.

\begin{definition}
Denote the category models of $\mathbb{T}_{bpalg}$ in $\Set$ by $\BPAlg$.
\end{definition}

\begin{remark}
    Note that we have an equivalence of categories
    \[
        \FinProb^{op} \simeq \BPAlg_f
    \]
    between the opposite category of finite probability spaces and the category of finite boolean probability algebras. We don't prove this in detail nor do we give the equivalence explicitly, since we won't use this fact anywhere in this paper.
\end{remark}

\subsection{The classifying topos proof}

We would like to show that the topos $\Set^{\FinProb^{op}}$ classifies $\mathbb{T}_{bpalg}$. More specifically, we will prove the equivalence
\[
\Set[\mathbb{T}_{bpalg}] \simeq \Set^{\FinProb^{op}} .
\]

The main difficulty is that we have we have found no way of knowing whether the topos on the left-hand-side has enough points. For example, Deligne's theorem is inapplicable, since the theory is not coherent.

Nevertheless, as a consistency check, one can verify that the models in $Set$ are equivalent by checking that
\[
    \BPAlg \simeq \Ind(\BPAlg_f) \simeq \Ind(\FinProb^{op}),
\]
but the proof of this fact is not short enough for us to provide it here.

We will prove the equivalence
\[
\Set[\mathbb{T}_{bpalg}] \simeq \Set^{\FinProb^{op}}
\]
by purely syntactical means. The functor in one direction will be given by an appropriate model of $\mathbb{T}_{bpalg}$ in $\Set^{\FinProb^{op}}$. The functor in the other direction will require us to work with the syntactic category $\mathcal{C}^{\mathbb{T}_{bpalg}}$ directly, and we will see that this will lead to us needing to work formally with geometric logic.

Before we begin, we explain the intuition behind this Morita-equivalence. The idea is that, while the left topos axiomatizes a boolean probability algebra, the topos on the right directly axiomatizes the \emph{partitions} of the boolean algebra into elements of prescribed measures. The object $\y U_{r_i}$ is supposed to represent elements $a_1,\cdots,a_n$ with measures $\mu(a_1) = r_1,\cdots,\mu(a_n) = r_n$. The maps $\y U_{s_i} \to \y U_{r_i}$ are supposed to represent unions of the partition $(s_i)$ into a coarser one.

With this intuition in mind, how should we go about defining the universal model defining the functor $\Set[\mathbb{T}_{bpalg}] \to \Set^{\FinProb^{op}}$? Essentially, we are asking how to construct an object of \emph{all} elements of a boolean algebra from the objects of elements with specified measures. Under this perspective, it becomes clear that we should define the underlying presheaf of the universal model $U$ to be
\[
\coprod_{r \in [0,1]} \y U_r \in \Set^{\FinProb^{op}} .
\]
One immediately sees that $U$ is just the powerset functor on the underlying set of a given probability space (a map $f$ between probability spaces gets sent to the map that takes the preimage of each subset).

\begin{lemma} \label{universal model}
$U$ is a model of the $\mathbb{T}_{bpalg}$. The sort $B$ is given by $U$. The relation symbols $B_r$ are given by $\y U_r$ and the boolean algebra operations given by the fact that at every $(X,\mu) \in \FinProb$, $U(X, \mu)$ is the powerset of $X$.

This yields a left-exact, colimit-preserving functor
\[
    \Set^{\FinProb^{op}} \xleftarrow{\Lan_{\y}(U)} \Set[\mathbb{T}_{bpalg}],
\]
where $U$ is the functor we get by identifying models of $\mathbb{T}_{bpalg}$ with functors whose domain is the syntactic category of $\mathbb{T}_{bpalg}$.
\end{lemma}
\begin{proof}
The last statement follows from the first part by using the universal property of classifying topoi.

We begin with some well-definedness remarks. The operations $\wedge, \vee$ and $\neg$ on $U$ can be defined pointwise by using the boolean algebra structure of powersets. Since the maps of finite probability spaces induce boolean algebra homomorphisms between the powersets, and these automatically preserve $\wedge, \vee \text{ and } \neg$, we get naturality of the operations. Similarly, the inclusions $\y U_r \to U$ are natural since morphisms in $\FinProb$ preserve measure.

Since we are looking at presheaves, all the colimit and limit conditions from the axioms can be checked pointwise, where they are trivial. More formally, we can use \cite{johnstone2002sketches2}, Corollary D.1.2.14, to get the fact that $U$ is a $\mathbb{T}_{bpalg}$-model in $\Set^{\FinProb^{op}}$.
\end{proof}

We proceed by formalizing our intuition about the representables in $\Set^{\FinProb^{op}}$ being partitions by constructing our candidate inverse functor.

We diverge from the notation in D.1 of \cite{johnstone2002sketches2} in that we will sometimes annotate a formulas with a set of variables containing all the variables used in the formula, as seen below. This is done so that when these formulas get combined with logical connectives, the big formula remains easy to read.

\begin{lemma} \label{inverse}
Let $f: (X,\mu) \to (Y, \nu)$ be an arrow in $\FinProb$. We define the functor $F: \FinProb \to \mathcal{C}^{\mathbb{T}_{bpalg}}$ on the object $(X,\mu)$ and on the function $f$ as follows: Let
\begin{align*}
\{\vec{a} . \phi_{X,\mu}^{\vec{a}}\} &:= \Bigg \{ \vec{a} . \bigwedge_{i \in X} B_{\mu(i)} a_i \wedge \bigwedge_{\substack{i,j \in X \\ i \neq j}} (a_i \wedge a_j = 0) \Bigg \} \\
[\theta_f^{\vec{a}, \vec{b}}] &:= \Bigg [ \bigwedge_{j \in Y} (b_j = \bigvee_{f i = j} a_i) \wedge \phi_{X,\mu}^{\vec{a}} \Bigg] ,
\end{align*}
and now define $F(X,\mu) := \{\vec{a} . \phi_{X,\mu}^{\vec{a}}\}$ and $F(f) := [\theta_f^{\vec{a}, \vec{b}}]$.

This yields a left-exact, colimit-preserving functor
\[
    \Lan_F: \Set^{\FinProb^{op}} \to \Set[\mathbb{T}_{bpalg}] .
\]
\end{lemma}

Readers paying close attention to the proof strategy may wonder why we even bother showing left-exactness of $\Lan_F$. After all, if at the end we show that we have an equivalence of categories, we would get left-exactness for free. But it turns out, that proving that our functors are inverses to one another(see \ref{classifying_presheaf}) essentially requires this left-exactness property. We have marked the place in the proof where we use it by a footnote.

\begin{proof}
    We need to show functoriality and then verify that $\Lan_F$ is left-exact. In order to check this left-exactness, we use theorem \Rmnum{7}.9.1 in \cite{maclane2012sheaves}(page 399). The cited theorem says that flat functors(defined as functors whose left Kan extension is left-exact) precisely correspond to the functors satisfying the straightforward list of properties listed in the definition of a filtering functor, which is defined in \Rmnum{7}.8.1 (page 394).

    The reader who actually reads the definition of a filtering functor will quickly notice that that definition can be checked in our case by showing that $F$ preserves the terminal object, binary multi-products and multi-equalizers in the sense checked below.

    \begin{enumerate}
        \item We show that $F$ preserves identities. We compute that
            \begin{align*}
                F(\id) &= [\theta^{\vec{a},\vec{b}}_{\id}] \\
                &= [\vec{a}=\vec{b}] \\
                &= \id,
            \end{align*}
            which shows our claim.
        \item We show that $F$ preserves composition. Given morphisms $f: \alpha \to \beta$, $g: \beta \to \gamma$ in $\FinProb$,
            \begin{align*}
                F(g)F(f) &= [\exists c . \theta^{\vec{a},\vec{c}}_f \wedge \theta^{\vec{c},\vec{b}}_g] \\
                &= [\theta^{\vec{a},\vec{b}}_{gf}] ,
            \end{align*}
            where the last equality follows from associativity and commutativity of the $\wedge$ operation and by additivity of the measure.
        \item We show that $F$ preserves the terminal object. We simply compute this:
            \begin{align*}
                F(*) &= \{a. B_1(a)\} \\
                &\cong \{[].\top \} ,
            \end{align*}
            where the last isomorphism follows from the fact we can geometrically prove that there is a unique element of measure $1$.
        \item We show that $F$ preserves multi-products. Concretely, let $\alpha$, $\beta$ be finite probability spaces. Then we have a polytope $P^2_{r_i,s_j}$ and for each of its elements $t_{ij}$ we have a cone
        \[\begin{tikzcd}
            & U_{t_{ij}} \arrow[ld, "p_{t_{ij}}"'] \arrow[rd, "q_{t_{ij}}"] &       \\
        \alpha &                                                               & \beta
        \end{tikzcd} .\]

        In order to show that this multi-product gets preserved, we need to show that the map
        \[
            \coprod_{t_{ij} \in P^2_{r_i,s_j}} \langle \theta^{\vec{a},\vec{b}}_{p_{t_{ij}}}, \theta^{\vec{a},\vec{b}}_{q_{t_{ij}}} \rangle : \coprod_{t_{ij} \in P^2_{r_i,s_j}} F U_{t_{ij}} \to F(\alpha) \times F(\beta)
        \]
        is an isomorphism. It is crucial that there does not only exist an isomorphism, but that it is this concrete isomorphism constructed by universal property. Indeed, for showing corresponding condition \Rmnum{7}.8.1 in \cite{maclane2012sheaves} we need to show that a specific family of morphisms is jointly epimorphic. And our claim here certainly implies that.

        We will show that this map is an isomorphism by computing this map as a formula. First, note that the inclusions
        \[
            FU_{t_{ij}} \to F(\alpha) \times F(\beta)
        \]
        can be computed (using D.1.4.2 in \cite{johnstone2002sketches2}) to be
        \[
            [ b_i = \bigvee_j a_{ij}, c_j = \bigvee_i a_{ij} ] :\{ \vec{a}. \phi^{\vec{a}}_{U_{t_{ij}}} \} \to \{ \vec{b},\vec{c} . \phi^{\vec{b}}_{\alpha} \wedge \phi^{\vec{c}}_{\beta} \} .
        \]
        By using the partition axiom multiple times, we notice that the formulas $\{ \vec{a} . \phi^{\vec{a}}_{U_{t_{ij}}}\}$ are disjoint (in the subobject lattice of $\{ \vec{a} . \top \}$). Thus the coproduct corresponds to union in that subobject lattice. Therefore by D.1.4.10 in \cite{johnstone2002sketches2} we can compute the required coproduct to be
        \[
            [ b_i = \bigvee_j a_{ij}, c_j = \bigvee_i a_{ij} ]: \{ \vec{a}.\bigvee_{t_{ij} \in P^2_{r_i,s_j} } \phi^{\vec{a}}_{U_{t_{ij}}} \} \to \{\vec{b},\vec{c} . \phi^{\vec{b}}_{\alpha} \wedge \phi^{\vec{c}}_{\beta}\},
        \]
        where the map is forced to be this map by universal property, since it commutes with the inclusions.

        Interestingly, we have shown that this map is simply a substitution. But we can give another substitution that can be checked to be an inverse of this map by using the boolean probability algebra axioms:
        \[
            [a_{ij} = b_i \wedge c_j]
        \]

        Thus we have shown that the required map is an isomorphism.

        \item  Let
        \begin{tikzcd}
        \alpha \arrow[r, "f", shift left] \arrow[r, "g"', shift right] & \beta
        \end{tikzcd}
        be a diagram in $\FinProb$. We show that $F$ preserves multi-equalizers. More specifically, if we have a family of forks
        \[
            \begin{tikzcd}
            \gamma_i \arrow[r, "h_i"] & \alpha \arrow[r, "f", shift left] \arrow[r, "g"', shift right] & \beta
            \end{tikzcd}
        \]
        forming a multi-equalizer, then we need to show that
        \[
            \coprod_i F(h_i) : \coprod_i F(\gamma_i) \to eq(\begin{tikzcd}
                F(\alpha) \arrow[r, "Ff", shift left] \arrow[r, "Fg"', shift right] & F(\beta)
                \end{tikzcd})
        \]
        is an isomorphism.

        We already know that multi-equalizers in $\FinProb$ are particularly simple. If $f = g$, then the family of forks consists of only the identity map $\id: \alpha \to \alpha$. In this case our claim becomes trivial.

        Now assume that $f \neq g$. Then the multi-equalizer is the empty family. Thus our claim becomes that
        \[
            eq(\begin{tikzcd}
                \{ \vec{a}.\phi^{\vec{a}}_{\alpha} \} \arrow[r, "{[\theta^{\vec{a},\vec{b}}_f]}", shift left] \arrow[r, "{[\theta^{\vec{a},\vec{b}}_g]}"', shift right] & \{ \vec{b}.\phi^{\vec{b}}_{\beta} \}
                \end{tikzcd})
        \]
        is the initial object $\{\vec{a}.\bot\}$. In fact, using D.1.4.2 in \cite{johnstone2002sketches2}, we can compute this equalizer to be $\{ \vec{a} . (\exists \vec{b}) \theta_f^{\vec{a},\vec{b}} \wedge \theta_g^{\vec{a},\vec{b}} \}$. This means that our objective is to show the sequent
        \[
            (\exists \vec{b}) \theta_f^{\vec{a},\vec{b}} \wedge \theta_g^{\vec{a},\vec{b}} \vdash_{\vec{a}} \bot .
        \]
        Since that $f \neq g$, we have a $k$ be such that $f(k) \neq g(k)$. Now we can simply show that this sequent holds.
        \begin{align*}
            (\exists \vec{b}) \theta_f^{\vec{a},\vec{b}} \wedge \theta_g^{\vec{a},\vec{b}}
            &\vdash_{\vec{a}} (\exists \vec{b}) \bigwedge_{j \in \beta} (b_j = \bigvee_{f i = j} a_i) \wedge \bigwedge_{j \in \beta} (b_j = \bigvee_{g i = j} a_i) \wedge \phi_{\alpha}^{\vec{a}} \\
            &\vdash_{\vec{a}} \bigwedge_{j \in \beta} \big ( \bigvee_{f i = j} a_i = \bigvee_{g i = j} a_i \big ) \wedge \phi_{\alpha}^{\vec{a}} \\
            &\vdash_{\vec{a}} \big ( \bigvee_{f i = f k} a_i = \bigvee_{g i = f k} a_i \big ) \wedge \phi_{\alpha}^{\vec{a}} \\
            &\vdash_{\vec{a}} \big ( \bigvee_{f i = f k} a_i \wedge a_k = \bigvee_{g i = f k} a_i \wedge a_k \big ) \wedge \phi_{\alpha}^{\vec{a}}\\
            &\vdash_{\vec{a}} (a_k = 0) \wedge \phi_{\alpha}^{\vec{a}} \\
            &\vdash_{\vec{a}} \bot
        \end{align*}
    \end{enumerate}
\end{proof}

\begin{theorem}[Classifying topos of $\mathbb{T}_{bpalg}$] \label{classifying_presheaf}
$\Lan_{\y}(U)$ and $\Lan_F$ define an equivalence of categories
\[
\Set^{\FinProb^{op}} \simeq \Set[\mathbb{T}_{bpalg}].
\]
\end{theorem}

We warn the reader that the necessary ideas defining the equivalence have already been introduced, and that this proof consists of highly technical and uninsightful property-checking.


\bibliographystyle{unsrt}
\bibliography{probability-topos}

\end{document}
